%
% mental-model-paper.tex
%
% This is the LaTex source file for a paper on the role the alignment of user and developer mental models play in the usability of skeuomorhpic interfaces.
%

%
% Use the standard article template.
%
\documentclass{article}

% The geometry package allows for easy page formatting.
\usepackage{geometry}
\geometry{letterpaper}

% Load up special logo commands.
\usepackage{doc}

% Package for formatting URLs.
\usepackage{url}

% Packages and definitions for graphics files.
\usepackage{graphicx}
\usepackage{epstopdf}
\DeclareGraphicsRule{.tif}{png}{.png}{`convert #1 `dirname #1`/`basename #1 .tif`.png}

%
% Set the title, author, and date.
%
\title{Skeuomorphism and Mental Models}
\author{Britain Southwick}
\date{October 30, 2012}

%
% The document proper.
%
\begin{document}

% Add the title section.
\maketitle

% Add an abstract.
\abstract{
Skeuomorphic interfaces have experienced a huge surge in popularity in recent years. This can be attributed, in part, to the ease with which they allow the developer to align their mental model of the software or app with that of the consumer. When done right and in the right context, skeuomorphic interfaces present an easy to use, flavorful, and inviting atmosphere that makes the software much more popular among consumers. However, there are certainly cases where a skeuomorphic interface can only serve to confuse and annoy, such as when it is implemented incorrectly or in the wrong type of software. 
}

% Add various lists on new pages.
\pagebreak
\tableofcontents

\pagebreak
\listoffigures

\pagebreak
\listoftables

% Start the paper on a new page.
\pagebreak

%
% Body text.
%
\section{Introduction}
\label{introduction}

A skeuomorphic interface is described by\cite{medialoot} as "A derivative object that retains ornamental design cues to a structure that was necessary in the original, even when not functionally necessary." In very general terms, a skeuomorphic interface is when an app that simulates a real world object is aesthetically designed to look like whatever it is simulating. Frequently, these designs have no function other than aesthetics. Apple Computers in particular is very fond of using skeuomorphic interfaces. 

Examples can be found all across their IOS, including the iCalendar, the Calculator, and the Notes app. Skeuomorphic interfaces are generally used for fun, creative apps and software such as painting programs or music creators. Their purpose is to make the app or software seem familiar and interesting to the consumer. With the proliferation of touch based user interfaces like the iPod and the iPad, skeuomorphic interfaces are becoming more and more widespread. The consumer response seems to be generally favorable (indicating why Apple pursues the design so judiciously), but among designers the response is much more divided. Some designers voice vehement opposition while others defend the virtues of skeuomorphism. This paper aims to examine these positions to come to a conclusion as to how effectively skeuomorphism aligns the mental models of the developers and the users.


\section{Background, Preliminary, and Related Work}


\section{Main Content Sections}



\subsection{Multiple Outline Levels}

\subsection{Tables and Figures}

\begin{table}
\centering
\begin{tabular}{|c|c|c|}\hline
Column 1 & Column 2 & Column 3 \\\hline\hline
a & b & c \\
d & e & f \\
g & h & i \\\hline
\end{tabular}

\caption{A sample table}
\label{table-sample}
\end{table}

\begin{figure}
\centering
\includegraphics[width=2in]{space.jpg} 

\caption{A sample figure}
\label{figure-sample}
\end{figure}

One very important thing to remember about how \LaTeX\ handles tables and figures by default: you don't have to worry about where they go exactly.  The general rule is that you insert them in the source after your first reference to them, and \LaTeX\ determines their final position.  It also makes decisions on how much page space to devote to them.  This all follows \LaTeX's overall theme of focusing on the content of your paper, and not its format.

Just so you can see a second table, Table~\ref{table-sample2} is provided.

\begin{table}
\centering
\begin{tabular}{|c|c|c|}\hline
Column 1 & Column 2 & Column 3 \\\hline\hline
a & b & c \\
d & e & f \\
g & h & i \\\hline
\end{tabular}

\caption{Another sample table}
\label{table-sample2}
\end{table}

\section{Another Section}

We're adding another section just so you can see how that looks.  Plus there are a few more \LaTeX\ features to illustrate.

\subsection{Bulleted and Numbered Lists}

\LaTeX\ is very good at providing clean lists.  Examples are shown below.

\begin{itemize}
\item Bulleted items come out properly indented and spaced, every time.

\begin{itemize}
\item Sub-bullets are a virtual no-brainer: just nest another \verb!itemize! block.
\item Note how the bullet character automatically changes too.
\end{itemize}

\item Just keep on adding \verb!\item!s\ldots

\item \ldots until you're done.
\end{itemize}

Numbered lists are almost identical, except that you specify \verb!enumerate! instead of \verb!itemize!.  List items are specified in exactly the same way (thus making it easy to change list types).

\begin{enumerate}
\item A list item
\item Another list item
\item A list item with multiple nested lists

\begin{itemize}
\item Nested lists can be of mixed types.
\item That's a lot of power and flexibility for the price of learning a handful of directives.

\begin{enumerate}
\item Like nested bullet lists, nested numbered lists also ``intelligently'' change their numbering schemes.
\item Meanwhile, all \emph{you} have to write is \verb!\item!.  \LaTeX\ does the rest.
\end{enumerate}
\end{itemize}

\item Back to your regularly scheduled list item

\end{enumerate}

\subsection{Subsection with Another Figure}

We may as well include a second figure also, shown in Figure~\ref{figure-sample2}.  The same image file is used, but note how it can be resized.  Again, observe how the positions of the tables and figures do not necessarily match their positions in the source file, reiterating the aforementioned \LaTeX\ functionality for deciding where these items go in the final document.  You provide an approximate location, and \LaTeX\ does the rest.

\begin{figure}
\centering
\includegraphics[width=1in]{space.jpg} 

\caption{Another sample figure}
\label{figure-sample2}
\end{figure}

\section{Conclusion}

Wrap up your paper with an ``executive summary'' of the paper itself, reiterating its subject and its major points.  If you want examples, just look at the conclusions from the literature.

% Generate the bibliography.
\bibliography{mental-model-paper}
\bibliographystyle{unsrt}

\end{document}
